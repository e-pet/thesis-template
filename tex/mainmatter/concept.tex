\chapter{Concept}
\label{chap:concept}

This chapter describes the novelty that you have created during the making of your thesis.
This novelty might be an experimental apparatus, a novel algorithm, or a thorough analysis of something.
You may here reuse the background information presented in the previous chapter.

There is a wealth of guides available on thesis writing; one that I'd like to point out here because of its conciseness and direct availability is \emph{Clarity in technical reporting}~\cite{Katzoff64}, an old yet still valid guide on the subject, originally created for internal use at NASA.

\section{Thesis writing strategies}

\subsection{Have a plan before you start}
Before starting to write any paragraph of full text in your thesis, you should know the contents of every chapter of your thesis to a high degree of detail.
Otherwise, you will find yourself rewriting and maybe even dumping large parts of what you wrote earlier.
Create a detailed outline of the story you want to tell in your thesis, and make it clear to yourself how each chapter contributes to this central story (also see below for general tips on structuring your thesis well).
It also helps to collect and create explanatory figures early on that can help you identify the concepts you need to explain in your thesis.

Once you have a very detailed draft of your thesis \emph{withoug having written any full sentences}, and you have discussed this draft in detail with your supervisor, doing the actual writing will be much, much easier than if you started right away.
It will also reduce the chance of your supervisor telling you to rewrite large parts of your thesis.
For a more detailed discussion of this writing strategy, also refer to \textcite{Carlis09}.

\subsection{Reproducible research}
\emph{Reproducible research} is a term that has gained a lot of attention recently.
It describes research that is easily reproducible by yourself, as well as others.
Note that many common research practices are not easily reproducible at all, e.g.
\begin{itemize}
\item Saving figures you somehow created interactively in \swname{Matlab} -- instead of saving a script file that reliably reproduces this figure.
\item Saving statistics calculated from measured data, or modified data (e.g. smoothed signals, or with outliers removed) -- instead of saving a script file that performs the desired operations.
\item Changing a results table in your report according to new results obtained during interactive analysis -- instead of saving a script file that automatically generates the data in the table, using the most current version of your software.
\end{itemize}
Keeping your work reproducible is not only useful for others, but also (and especially) for yourself: It makes sure that all parts of your work (data, analysis, report, etc.) are always in sync.
Furthermore, you can, e.g., easily update all figures in your thesis a week before the deadline -- which may be a nightmare if you created all of these figures manually and cannote reproduce them easily.
A good introduction to the steps required to keep your work reproducible can be found at \url{http://kbroman.org/steps2rr/}.
I do recommend checking it out right now!

\section{Structure}
The general structure of your text depends on the particular subject and your personal preferences.
A reader following your text should at each point in your thesis feel like you are telling him a consistent story.
In particular, ask yourself the following questions:
\begin{itemize}
\item What is your main contribution to the scientific community? 
\item Which concepts and background knowledge do you need to provide for the reader to be able to understand your contribution and its relevance?
\item Which would be a sensible and natural ordering of the concepts that you need to explain?
\end{itemize}
Always keep in mind that the structure of your text should feel natural to a reader, not leaving him or her wondering what on earth this new topic has to do with anything he's read so far.

Start your scientific text with an introduction, that
\begin{itemize}
    \item introduces the subject,
    \item specifies the topic,
    \item reflects on the problem that you are going to consider,
    \item defines the purpose of the work,
    \item explains the line of reasoning
    \item sketches the structure of the work.
\end{itemize}

Keep in mind that there are some readers that only read the introduction and conclusion of your work and base their decision on whether the work bears any interest for them only on these parts. 
To make that decision, they need to get all relevant information from those two chapters. 
Hint: Look at other work with a focus on that question.

For the main part of your work, there are no general rules. 
Among others, the order of your chapters should depend on whether your work has a rather theoretic, methodical or experimental focus. 
The length of each chapter does not necessarily need to be proportional to the amount of time you have spent on solving the respective problems. 
Sometimes it takes one week to debug a piece of code, which nevertheless should not be explained excessively.

Your work concludes with a summary of your results.
Have a look at your introduction: how you have specified the problem there? Do your results solve the problem?
Do not present any further results here that have been not presented in the main part. 
As such, always clearly separate the presentation and the discussion of results.

Finally, you end with an outlook that points out open questions. 
What should be further analyzed and what are possible follow-up projects?
Do not be afraid to point out questions that came to your mind during your research, but you did not have time to properly answer. 
A good thesis may raise more questions than it clarifies.


\section{Language}
For a very good and readable (and cheap) guide on improving your writing skills, refer to \textcite{Zinsser76}: \emph{On Writing Well}.
Every minute you spend reading this book will be well worth its time.

\subsection{Correctness}
Scientific writing serves a single purpose: to transport information as efficiently as possible.
To achieve this, your writing should be as accurate as possible. 
Do not use colloquial language.
Respect the rules of grammar, spelling and punctuation; phrases or words used in the wrong context can lead to misunderstandings or may be hard to understand.
It should be clear that the results of your work, e.g., experiments, must be documented precisely and correctly, even though they might have had unexpected outcomes.
Otherwise you do not only cause harm to you but any further research.


\subsection{Comprehensibility}
Correctness does not imply comprehensibility.
Look at your text from the readers point of view: Consider his or her position, previous knowledge and attitude. Formulate as precisely as possible but not more complicated than necessary. 
Therefore,:
\begin{itemize}
    \item choose words, that are known;
    \item if you have to use words that are probably unknown, use them such that their meaning can be deduced from the context, or explain or define them;
    \item do not construct deeply nested sentences.
\end{itemize}

To achieve clarity and conciseness in your writing, you may find the following method very useful (I use this whenever I write \emph{any} text).
For any paragraph you have written, ask yourself the following questions:
\begin{itemize}
	\item Which information or logical argument do I want to convey to my reader using this paragraph? Which are the different steps of this argument, and in which order to they make sense?
	\item Does the paragraph have a clear focus on a single subject, or do you write about a number of different things? (If the latter is the case, separate the contents into multiple paragraphs.)	
	\item Are there logical \glqq gaps\grqq{} in the paragraph, i.e., are there steps missing to be able to understand the described argument? (If yes, insert them.)	
	\item Do I currently convey this information in the most effective way? Are there phrases that could be shortened or removed without hurting the paragraph's main message? (If yes, do so.)
	\item Does the paragraph \glqq flow\grqq{} naturally from one phrase to the other, following a logical progression? (If not, think again about the steps of the logical argument you are trying to give, and try to emphasize the logical connections between consecutive phrases.)
\end{itemize}

Many example applications of this method can be found in the book of Zinsser recommended above.
While it takes a while to get used to this way of thinking about writing, I find this method tremendously useful for communicating complex subjects effectively.
Note that the same questions can also be asked both on a smaller (single sentence) as well as larger scale (whole section or chapter).
Doing so can really help to strip down your text to the absolutely necessary minimum required to convey the main results of your work, which will greatly improve the clarity and conciseness of your thesis.

\section{Preparing a scientific presentation}
Generally, the same guiding principles that apply to writing a scientific report also apply to the preparation of a scientific presentation: A meaningful structure, clarity, conciseness and correct presentation of contents, including correct referencing of previous works of other scientists.
However, in a presentation, the focus should be strongly shifted away from using a lot of text, towards the display of informative figures and diagrams that visually support your oral presentation of concepts and results.

A recent trend in scientific presentation styles has been to move away from the classical \glqq Introduction - Methods - Results - Conclusion\grqq{} structure of presentations, and more towards what is called an \glqq Assertion - Evidence\grqq{} structure.
In this style, slides are usually composed of a title line that displays a meaningful message you would like to convey to your audience (\glqq Respiratory Mechanics can be modeled as Electrical Circuits\grqq{}), instead of a classical caption (\glqq Modelling of Respiratory Mechanics\grqq{}), and supporting visual evidence (i.e., a mechanical and an electrical model of respiratory mechanics), instead of bullet point lists.
All further information required to understand the slide is  presented orally by the speaker.
Scientific studies have shown this style of slides to be more effective in conveying information to the audience than classically designed slides.
For more information on this presentation style (and pointers towards the mentioned studies), refer to \url{http://www.assertion-evidence.com/}.

The following is just a further quick link list compiled to help in creating good scientific presentations.
\begin{itemize}
\item \url{http://www.the-scientist.com/?articles.view/articleNo/28818/title/Pimp-your-PowerPoint/}
\item \url{http://www.northwestern.edu/climb/resources/oral-communication-skills/creating-a-presentation.html}
\item \url{http://www.nextscientist.com/improve-presentation-skills-of-phd-students/}
\end{itemize}


%%%%% Emacs-related stuff
%%% Local Variables: 
%%% mode: latex
%%% TeX-master: "../../main"
%%% End: 
