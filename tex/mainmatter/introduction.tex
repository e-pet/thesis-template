\chapter{Introduction}
\label{chap:intro}

You've just started your project work, your bachelor or master's thesis. 
Maybe you've already achieved some results like experiments, algorithms, methods or you've made some handwritten notes. 
Now you wonder how to put that on paper?

This document is both a manual on how to write a thesis, as well as a template that you can use for your own thesis.

When you've read this document and looked through the accompanying \LaTeX-files it is generated from, you should be able to produce your own scientific work, in a clear and visually appealing form.

\fxnote{This is a FixMe Note that tells you to complete the introduction of your thesis.
You can use FixMe to help you organizing your thesis writing process, if you like. 
For more information on how to use it, see the FiXme manual~\cite{fixme}}

\section{Previous Works}
This section should contain a quick overview of the previous results of other researchers or students relevant to your thesis project.

\section{Thesis outline}
This work is structured in two sections. 
\Cref{chap:materials-and-methods} describes how to produce a \LaTeX-file using some illustrative examples. 
\Cref{chap:concept} gives a short introduction to scientific writing techniques.


%%%%% Emacs-related stuff
%%% Local Variables: 
%%% mode: latex
%%% TeX-master: "../../main"
%%% End: 
