\chapter{Introduction}
\label{chap:intro}

You've just started your project work, your bachelor or master's thesis. 
Maybe you've already achieved some results like experiments, algorithms, methods or you've made some handwritten notes. 
Now you wonder how to put that on paper?
This document is both a manual on how to write a thesis as well as a template that you can use for your own thesis.
Once you've read this document and checked out the accompanying \LaTeX-files from which it is generated, you should be able to produce your own scientific work, in a professional and visually appealing form.
If you have any feedback (positive or negative) regarding this document, please \href{mailto:eike.petersen@uni-luebeck.de}{let me know}!
\\

\noindent\emph{Eike Petersen, March 2019}\\
\noindent(\href{mailto:eike.petersen@uni-luebeck.de}{eike.petersen@uni-luebeck.de})

\fxnote{This is a FixMe Note that tells you to complete the introduction of your thesis.
To see how this is achieved, see \texttt{introduction.tex}.
You can use FixMe to help you organizing your thesis writing process, if you like. 
For more information see the FiXme manual~\cite{fixme}.}

\section{Previous Works}
This section should contain a quick overview of the previous results of other researchers or students relevant to your thesis project.
It serves to clearly position the results of your thesis in relation to other works, and to clarify the novelty of your work in comparison to previous approaches.
This is relevant because it communicates to the reader why your work is valuable and useful to the scientific community!

Normally, this section should span between a half and one page, and you should provide many references here.
Note that this is not an easy section to write, because it requires you to have a solid understanding of the precise differences between various existing approaches, and how your method relates to them.
To be able to do this, you should spend some time searching for, reading, and understanding papers and textbooks relevant to your thesis.

\section{Thesis outline}
This work is structured in two sections. 
\Cref{chap:materials-and-methods} describes how to produce a \LaTeX-file using some illustrative examples. 
\Cref{chap:concept} gives a short introduction to scientific writing techniques.


%%%%% Emacs-related stuff
%%% Local Variables: 
%%% mode: latex
%%% TeX-master: "../../main"
%%% End: 
