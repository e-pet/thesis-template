\chapter{Materials and Methods}
\label{chap:materials-and-methods}

This chapter deals with background information relevant for your thesis, including physiological background, existing research on the topic, and mathematical and other preliminaries required to understand the novel concepts presented in the following chapter.


\section{Getting started with Latex}
\label{sec:getting-started-with}

To get started with Latex, you need...
\begin{itemize}
\item a tool that generates a PDF file out of a bunch of *.tex and *.bib files.
Under Windows, this is typically \swname{MikTex} (\url{http://miktex.org/download}).
\item a text editor. This can be as simple as \swname{Notepad++}(\url{https://notepad-plus-plus.org/}), but many would recommend an IDE that provides further convenience features. \swname{TexMaker}(\url{http://www.heise.de/download/texmaker.html}) and \swname{TexStudio}(\url{http://www.texstudio.org/}) are two well-known examples.
\end{itemize}
If you have these two installed on your PC, you're ready to go!

There is a wealth of references and tutorials on the internet that deal with Latex.
What follows is a small list compiled based on my personal preferences.
\begin{itemize}
\item ShareLatex currently provides - to my taste - the best introduction and reference on a number of Latex-related topics: \url{https://www.sharelatex.com/learn}.
\item Latex-Wikibooks often prove useful if you're really looking for a reference of the available symbols, e.g. \url{https://en.wikibooks.org/wiki/LaTeX/Mathematics} or \url{https://en.wikibooks.org/wiki/LaTeX/Tables}.
\item There is a neat little online tool available, which provides users with hints on available Latex commands based on the user's drawing of the desired symbol: \url{http://detexify.kirelabs.org/classify.html}.
\item Malte Schmitz from the Uni Lübeck also provides good introductory material in German: \url{http://www.mlte.de/layout}.
\end{itemize}

\section{Thesis writing strategies}

\subsection{Have a plan before you start}
Before starting to write any paragraph of full text in your thesis, you should now to a high degree of detail the contents of every chapter of your thesis.
Otherwise, you will find yourself rewriting and maybe even dumping large parts of what you wrote earlier.
Ideally, create a detailed outline of the story you want to tell in your thesis, and make it clear to yourself how each chapter contributes to this central story.
It also helps to collect and create explanatory figures early on that can help you identify the concepts you need to explain in your thesis.

\subsection{Reproducible research}
\emph{Reproducible research} is a term that has gained a lot of attention recently.
It describes research that is easily reproducible by yourself, as well as others.
Note that many common research practices are not easily reproducible at all, e.g.
\begin{itemize}
\item Saving figures you created somehow \swname{Matlab}, without saving a script file that can reproduce this figure.
\item Somehow modifying measured data (e.g. performing a smoothing operation), without saving anywhere the details of what you did.
\item Changing a results table in your report according to new results - but a week later you can't recall how you came to these results.
\end{itemize}
Keeping your work reproducible is not only useful for others, but also (and especially) for yourself: It makes sure that all parts of your work (data, analysis, report, etc.) are always in sync.
Furthermore, you can, e.g., easily update all figures in your thesis a week before the deadline - which may be a nightmare if you created all of these figures manually and cannote reproduce them easily.
A good introduction to the steps required to keep your work reproducible can be found at \url{http://kbroman.org/steps2rr/}.

\section{Preparing a scientific presentation}
The following is just a quick list compiled to help in creating good scientific presentations.
\begin{itemize}
\item \url{http://www.the-scientist.com/?articles.view/articleNo/28818/title/Pimp-your-PowerPoint/}
\item \url{http://www.northwestern.edu/climb/resources/oral-communication-skills/creating-a-presentation.html}
\item \url{http://www.nextscientist.com/improve-presentation-skills-of-phd-students/}
\end{itemize}




%%%%% Emacs-related stuff
%%% Local Variables: 
%%% mode: latex
%%% TeX-master: "../../main"
%%% End: 
